\chapter{Tests}
Diese Testdokumentation wurde erstellt, um die Herangehensweise, Durchführung sowie die Ergebnisse unseres Testprozesses festzuhalten.

\section{Testplan}

\subsection{Ziele}
Ziel unseres Testprozesses ist es garantieren zu können, dass die in diesem Projekt geschaffene Software unter den von uns festgelegten Vorraussetzungen annähernd bis vollständig fehlerfrei und mit möglichst guter Performance betrieben werden kann. Auffälligkeiten sowie nach dem Testprozess bekannte und unbehobene Fehler sollen am Ende des Testprozesses dokumentiert sein.

\subsection{Rahmenbedingungen}
Grundsätzlich wurde während der Entwicklung der Anwendung stets darauf geachtet, dass die jeweils neu implementierten Features einwandfrei funktionieren und auch, dass durch die Implementierung jener Features keine der zuvor vorhandenen Teile beschädigt werden. Dennoch haben wir in unserer Projektplanung eine gesonderte Testphase geplant, bei der wir im Zeitraum von zwei Wochen alle nötigen Schritte abschließen möchten, um die von uns erstellte Anwendung ausgiebig zu testen.



 

\subsection{Teststrategie}
Um unser Testziel zu erreichen greifen wir auf verschiedene Testmethoden zurück. Da die Testphase sowohl durch einen kurzen Zeitraum, als auch die Anzahl der Tester eingeschränkt ist, müssen wir diese Ressourcen bestmöglich nutzen. Nach längeren Diskussionen innerhalb des Entwicklungsteams haben wir uns dazu entschlossen, auf eine Kombination von automatisierten Unit-Tests, manuellen System- und UI-Tests, sowie Last-Test zu setzen. Auf diese Weise decken wir beim Testen nicht nur funktionale sondern auch nichtfunktionale Anforderungen der Software ab.

Welche Methodik bei den einzelnen Teilen der Anwendung verwendet wurde wird in der folgenden Tabelle dargestellt.

\begin{table}[]
	\begin{tabular}{|l|l|}
		\cline{1-2}
		\textbf{Testobjekt} & \textbf{Art des Testens} \\ \cline{1-2}
		Front-End           & \begin{tabular}[c]{@{}l@{}}Unit-Tests,\\ Manuelle Tests\end{tabular} \\ \cline{1-2}
		Back-End            & \begin{tabular}[c]{@{}l@{}}Unit-Tests.\\ Manuelle Tests\end{tabular} \\ \cline{1-2}
		Gesamtsystem        & Manuelle Tests \\ \cline{1-2}
	\end{tabular}
\end{table}

Trotz dessen, dass wir eine eigene Testphase geplant haben, ist es uns wichtig über den gesamten Entwicklungsprozess der Software für eine stets einwandfrei lauffähige Anwendung zu sorgen. Dies entspricht nicht nur unserem agilen Softwareentwicklungsprozess nach Scrum, sondern erleichtert auch die gemeinsame Arbeit durch mehrere Entwickler jeweils an Front-End sowie Back-End. Um dies gewährleisten zu können, haben wir abseits der Testphase jedes neu implementierte Feature sowie die Auswirkungen der Implementierung auf den Rest der Anwendung manuell getestet.

\section{Testen des Front-Ends}

\subsection{Unit-Tests}
Bei unserem Front-End sind wir zu dem Schluss gekommen, dass ein automatisiertes Testen nur bedingt sinnvoll ist. Ein großer Teil der Implementierungen dort bezieht sich rein auf die Darstellung der vom Back-End erhaltenen Daten im Webbrowser, oder um das Beschaffen und Versenden eben dieser Daten. Ein automatiertes Testen der Weboberfläche ist dabei überproportional aufwändig und in unserem Falle in den meisten Fällen nicht sinnvoll, da es sich vor allem um statische Inhalte oder um Video- beziehungsweise Bildinhalte handelt. Außerdem muss beim Testen einer Weboberfläche auf Faktoren wie Browserkompatibilität geachtet werden, was durch manuelles Testen besser umsetzbar ist. Nichts desto trotz wurde für jede Komponente des Front-Ends ein eigener Unit-Test erstellt, der die vollständige Erzeugung eben dieser Komponente simuliert und Testet. Dabei werden für die Komponente erforderliche Abhängigkeiten durch Mock-Objekte ersetzt, um ein unabhängiges Testen zu ermöglichen.

Standardgemäß verwenden wir beim automatisierten Testen unseres Angular-Front-Ends das Testframework Karma. Dieses ist bereits beim Erzeugen eines neuen Angular-Projektes per Angular-\acs{CLI} integriert und vorkonfiguriert. 

\subsubsection{Ausführen der Unit-Tests}
Nachdem das Projekt korrekt auf die in 23123 Dargestellte Art und Weise installiert wurde und lauffähig ist, können die automatisierten Tests durch das Aufrufen eines Konsolenbefehls gestartet werden. Dazu muss im Projektordner ein Terminal geöffnet werden und der Befehl "ng test" ausgeführt werden.

\subsubsection{Ergebnisse der Unit-Tests}
HIER SCREENSHOT VON UNITTEST ERGEBNISSEN EINFÜGEN


\subsection{Manuelle Tests}
Beim manuellen testen handelt es sich um einen Testprozess, bei dem der Tester ohne die Verwendung von Automatisierungstools vorgeht. Dabei können durch die systematische Verwendung der Software und das Nutzen von Diagnosetools oft Fehler aufgedeckt werden, die etwa bei Unit-Tests häufig nicht gefunden werden. Insbesondere Benutzeroberflächen können auf diese Weise unkompliziert getestet werden.

Im folgenden wird tabellarisch festgehalten, welche Aktionen getestet wurden und von welcher Ausgangssituation aus getestet wurde. Alle Tests wurden in den beiden Browsern Google Chrome (64-Bit Version 71.0.3578.98 Offizieller Build) und Mozilla Firefox (64-Bit Version 63.0.1 Offizieller Build) auf einem mit Windows 10 betriebenem Laptop mit einer Auflösung von 1920x1080 durchgeführt. 

\subsubsection{Ergebnisse der Manuellen Tests}
Vorraussetzung für alle Tests ist selbsterklärend, dass Front-End sowie Back-End korrekt installiert und gestartet sind. Zudem sind alle Einstellungen sinnvoll gewählt. Das bedeutet beispielsweise, dass ein funktionierender MJPEG-Stream hinterlegt ist.

Folgende Einstellungen waren bei den folgenden manuellen Tests vorhanden:

\begin{table}[]
	\begin{tabular}{|l|l|}
		\cline{1-2}
		\textbf{Einstellung} & \textbf{Wert} \\ \hline
		
		Sourrce.Livestrean-\acs{URL} & https://webcam1.lpl.org/axis-cgi/mjpg/video.cgi \\ \hline
		
		Maximum Clip Length & 10 \\ \hline
		
		Maximum Clip Count & 20 \\ \hline
		
		Maximum Clip Storage & 1024 \\ \hline
	\end{tabular}
\end{table}

Bedeutung der Spalte C*: Der Test wurde in der zuvor genannten Version von Google Chrome erfolgreich durchgeführt.

Bedeutung der Spalte F*: Der Test wurde in der zuvor genannten Version von Mozilla Firefox erfolgreich durchgeführt.

\newcounter{TestNumber}
\begin{longtable}{| p{.02\textwidth} | p{.08\textwidth} | p{.25\textwidth} | p{.25\textwidth} | p{.25\textwidth} | p{.025\textwidth} | p{.025\textwidth} |}
	\hline
	
	\textbf{\#} & \textbf{Kom-po-nen-te} & \textbf{Vorraussetz-ungen} & \textbf{Aktion} & \textbf{Erwartetes Ergebnis} & \textbf{C*} & \textbf{F*} \\ \hline
	
	\stepcounter{TestNumber}\arabic{TestNumber} & Login & Der Anwender befindet sich auf der Login-Seite und ist demnach nicht eingeloggt. & Der Anwender gibt beim Einloggen den richtigen Usernamen (user) und das richtige Passwort (geheim) ein. & Der Anwender wird auf die Hauptseite der Anwendung weitergeleitet und ist	korrekt eingeloggt. & X & X \\ \hline
	
	\stepcounter{TestNumber}\arabic{TestNumber} & Login & Der Anwender befindet sich auf der Login-Seite und ist demnach nicht
	eingeloggt. & Der Anwender gibt beim
	Einloggen einen falschen Usernamen (Verwendet: test) und das richtige Passwort (geheim) ein. & Rechts neben dem Login-Button erscheint eine Nachricht (Login failed) in roter Schrift. & X & X \\ \hline
	
	\stepcounter{TestNumber}\arabic{TestNumber} & Login & Der Anwender befindet sich auf der Login-Seite und ist demnach nicht
	eingeloggt. Das Back-End ist nicht erreichbar. & Der Anwender versucht sich einzuloggen. & Rechts neben dem Login-Button erscheint eine Nachricht (Login failed) in roter Schrift. & X & X \\ \hline
	
	\stepcounter{TestNumber}\arabic{TestNumber} & Login & Der Anwender befindet sich auf der Login-Seite und ist demnach nicht eingeloggt. & Der Anwender gibt beim
	Einloggen den richtigen Usernamen (user) und ein falsches Passwort ein. (Verwendet: test) & Rechts neben dem Login-Button  erscheint eine Nachricht (Login failed) in roter Schrift. & X & X \\ \hline
	
	\stepcounter{TestNumber}\arabic{TestNumber} & Login & Der Anwender befindet sich auf der Login-Seite	und ist demnach nicht
	eingeloggt. & Der Anwender gibt beim Einloggen sowohl einen falschen Usernamen (Verwendet: test1) als auch ein falsches Passwort (Verwendet: test2) ein. & Rechts neben dem Login-Button erscheint eine Nachricht (Login failed) in roter Schrift. & X & X \\ \hline
	
	\stepcounter{TestNumber}\arabic{TestNumber} & Login & Der Anwender befindet sich auf der Login-Seite und ist demnach nicht eingeloggt. & Der Anwender gibt beim Einloggen einen falschen Usernamen (Verwendet: test) und das richtige Passwort (geheim) ein. & Zwischen dem Absenden der Logindaten und dem Empfangen einer Antwort durch das Back-End wird rechts neben dem Login-Button eine Ladeanimation angezeigt. & X & X \\ \hline
	
	\stepcounter{TestNumber}\arabic{TestNumber} & Login & Der Anwender befindet sich auf der Login-Seite und ist demnach nicht eingeloggt. & Der Anwender gibt beim Einloggen den richtigen Usernamen (user) und das richtige Passwort (geheim) ein. & Zwischen dem Absenden	der Logindaten und dem Empfangen einer Antwort
	durch das Back-End wird	rechts neben dem Login-
	Button eine Ladeanimation angezeigt.  & X & X \\ \hline
	
	\stepcounter{TestNumber}\arabic{TestNumber} & Header & Der Anwender befindet sich auf der Login-Seite
	und ist demnach nicht eingeloggt. & Der Anwender klickt auf das
	PIPCO-Logo auf der linken Seite des Headers. & Die Webseite wird neu geladen. & X & X \\ \hline
	
	\stepcounter{TestNumber}\arabic{TestNumber} & Header & Der Anwender befindet sich auf der Settings-Seite und ist demnach	bereits eingeloggt. & Der Anwender klickt auf das PIPCO-Logo auf der linken Seite des Headers. & Die Webseite wird neu geladen. Der Anwender ist
	nicht länger eingeloggt und	wird daher auf die Login-Seite weitergeleitet. & X & X \\ \hline
	
	\stepcounter{TestNumber}\arabic{TestNumber} & Header & & Der Anwender hovert mit dem Cursor über das PIPCO-Logo auf der linken Siete des Headers. & Ein Tooltip (Refresh Page) wird neben dem Cursor angezeigt. Der Cursor ändert sein Styling zu Pointer. & X & X \\ \hline
	
	\stepcounter{TestNumber}\arabic{TestNumber} & Header & Der Anwender ist korrekt eingeloggt und befindet sich auf der Hauptseite & & Auf der rechten Seite des headers befinden sich ein Settings-Button sowie ein Logout-Button (in dieser Reihenfolge) & X & X \\ \hline
	
	\stepcounter{TestNumber}\arabic{TestNumber} & Header & Der Anwender ist korrekt eingeloggt und befindet sich auf der Hauptseite. & Der Anwender hovert mit dem Cursor über den Settings-Button auf der rechten Seite des Headers. & Ein Tooltip (Settings) wird neben dem Cursor angezeigt. Der Cursor ändert sein Styling zu Pointer. & X & X \\ \hline
	
	\stepcounter{TestNumber}\arabic{TestNumber} & Header & Der Anwender ist korrekt eingeloggt und befindet sich auf der Hauptseite. & Der Anwender hovert mit dem Cursor über den Logout-Button auf der rechten Seite des Headers. & Ein Tooltip (Logout) wird neben dem Cursor angezeigt. Der Cursor ändert sein Styling zu Pointer. & X & X \\ \hline
	
	\stepcounter{TestNumber}\arabic{TestNumber} & Header & Der Anwender ist korrekt eingeloggt und befinet sich auf der Hauptseite. & Der Anwender klickt auf den Settings-Button auf der rechten Seite des Headers. & Der Anwender wird auf die Settings-Seite weitergeleitet. & X & X \\ \hline

	\stepcounter{TestNumber}\arabic{TestNumber} & Header & Der Anwender ist korrekt eingeloggt und befinet sich auf der Hauptseite. & Der Anwender klickt auf den Logout-Button auf der rechten Seite des Headers. & Der Anwender wird korrekt ausgeloggt und auf die Login-Seite weitergeleitet. & X & X \\ \hline

	\stepcounter{TestNumber}\arabic{TestNumber} & Header & Der Anwender ist korrekt eingeloggt und befinet sich auf der Settings-Seite. & & Auf der rechten Seite des Headers befinden sich ein Home-Button sowie ein Logout-Button (in dieser Reihenfolge) & X & X \\ \hline
	
	\stepcounter{TestNumber}\arabic{TestNumber} & Header & Der Anwender ist korrekt eingeloggt und befindet sich auf der Settings-Seite. & Der Anwender klickt auf den Home-Button auf der rechten Seite des Headers. & Der Anwender wird auf die Hauptseite weitergeleitet & X & X \\ \hline
	
	\stepcounter{TestNumber}\arabic{TestNumber} & Header & Der Anwender ist korrekt eingeloggt und befindet sich auf der Settings-Seite. & Der Anwender klickt auf den Logout-Button auf der rechten Seite des Headers. & Settings-Button sowie Logout-Button im Header sind nicht mehr da. & X & X \\ \hline
	
	\stepcounter{TestNumber}\arabic{TestNumber} & Header & Der Anwender ist korrekt eingeloggt und befindet sich auf der Settings-Seite. & Der Anwender hovert mit dem Cursor über den Home-Button auf der rechten Seite des Headers. & Ein Tooltip (Home) wird neben dem Cursor angezeigt. Der Cursor ändert sein Styling zu Pointer. & X & X \\ \hline

	\stepcounter{TestNumber}\arabic{TestNumber} & Video & Der Anwender ist korrekt eingeloggt und befindet sich auf der Hauptseite. & & Der in den Einstellungen hinterlegte MJPEG-Stream wird angezeigt. & X & X \\ \hline	
	
	\stepcounter{TestNumber}\arabic{TestNumber} & Video & Der Anwender ist korrekt eingeloggt und befindet sich auf der Hauptseite. & & Über dem MJPEG-Stream wird eine Überschrift dargestellt (Currentyl Watching: IP Camera Live Stream) & X & X \\ \hline	
	
	\stepcounter{TestNumber}\arabic{TestNumber} & Video & Der Anwender ist korrekt eingeloggt und befindet sich auf der Hauptseite. & Aus der Event-Log-Komponente wird das Thumbnail einer Aufnahme angeklickt. & Der MJPEG-Stream wird durch eine Video-Wiedergabe des ausgewählten Clips ersetzt. & X & X \\ \hline
	
	\stepcounter{TestNumber}\arabic{TestNumber} & Video & Der Anwender ist korrekt eingeloggt und befindet sich auf der Hauptseite. & Aus der Event-Log-Komponente wird das Thumbnail einer Aufnahme angeklickt. & Der Titel über der Clip-Wiedergabe ändert sich (Currently Watching: Motion Detection Clip) & X & X \\ \hline
	
	\stepcounter{TestNumber}\arabic{TestNumber} & Video & Der Anwender ist korrekt eingeloggt und befindet sich auf der Hauptseite. & Aus der Event-Log-Komponente wird das Thumbnail einer Aufnahme angeklickt. & Neben dem Titel über der Clip-Wiedergabe erscheint rechts ein Button (RETURN TO LIVESTREAM) & X & X \\ \hline
	
	\stepcounter{TestNumber}\arabic{TestNumber} & Video & Der Anwender ist korrekt eingeloggt und befindet sich auf der Hauptseite. Über die Event-Log-Komponente wurde die Wiedergabe eines Clips gestartet. & Es wird auf den Return-Button (RETURN TO LIVESTREAM) rechts oben in der Komponente geklickt. & Die Clip-Wiedergabe wird durch den MJPEG-Stream ersetzt. & X & X \\ \hline
	
	\stepcounter{TestNumber}\arabic{TestNumber} & Video & Der Anwender ist korrekt eingeloggt und befindet sich auf der Hauptseite. Über die Event-Log-Komponente wurde die Wiedergabe eines Clips gestartet. & Es wird auf den Return-Button (RETURN TO LIVESTREAM) rechts oben in der Komponente geklickt. & Die Überschrift über der Wiedergabe wird zurückgesetzt (Currentyl Watching: IP Camera Live Stream) & X & X \\ \hline
	
	\stepcounter{TestNumber}\arabic{TestNumber} & Video & Der Anwender ist korrekt eingeloggt und befindet sich auf der Hauptseite. Über die Event-Log-Komponente wurde die Wiedergabe eines Clips gestartet. & Es wird auf den Return-Button (RETURN TO LIVESTREAM) rechts oben in der Komponente geklickt. & Der eben betätigte Return-Button verschwindet. & X & X \\ \hline
	
	\caption{Manuelle Front-End-Tests}
	\label{tab:manuelle_front_end_tests}
\end{longtable}
