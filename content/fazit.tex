\chapter{Fazit}
Das Projekt ist gut verlaufen und es gab nur wenige Probleme. Alle Termine konnten eingehalten werden und die geschaffene Software ist in einem vorzeigbaren Zustand. Trotzdem würde ich die Software noch nicht als Produktionsreif betrachten, da es noch viel Potenzial für neue Funktionen und Verbesserungen gibt.
Im Verlauf des Projektes konnten wir uns einiges, neues Wissen über verschiedene Technologien aneignen. Aber auch vorhandenes Wissen konnten wir, da wir viele Technische und Designentscheidungen selbst treffen mussten, sinnvoll einbringen. Vor Allem die Benutzung von OpenCV war interessant, da wir unser Wissen aus dem Modul Digitale Bildverarbeitung vertiefen und anwenden konnten. Auch durch die Benutzung von Angular war eine gute Entscheidung und eine große Hilfe beim erstellen des Frontends. Das Framework, war einigen von uns bereits vorher bekannt und an sich vereinfacht es viele Dinge. 
Die Wahl der Sprache Python für das OpenCV wurde uns etwas zum Verhängnis, da der Rechenaufwand für die Bildverarbeitung doch erheblich ist und dieser mit C++ vermutlich weniger ins Gewicht gefallen wäre. 
Im Team gab es im Projektverlauf keinerlei Spannungen und es herrschte stets ein gutes Klima. Die Aufgaben wurden fair aufgeteilt jedoch halfen wir uns oft untereinander, wenn es Probleme gab.
Abschließend kann Man sagen, dass das Projekt ein Erfolg war.