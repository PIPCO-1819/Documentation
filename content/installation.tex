\chapter{Installation}
\label{chapter_installation}

\section{System}
Die Installationsanweisungen wurden auf einem Ubuntu Server der Version 18.10 durchgeführt.\\
Hierzu wurde ein bereits installiertes Image für VirtualBox von \\ \texttt{https://www.osboxes.org/ubuntu-server/} verwendet.
  
\section{Backend}
Für das Backend muss OpenCV, sowie der Flask-Webserver mit allen notwendigen Modulen installiert werden.
Die Lightweight Installation von OpenCV, welche einfach mit pip installiert werden kann, enthält nicht den passenden Encoder für mp4, weshalb der aktuelle Stand selbst geladen und compiliert werden muss.
Hierzu kann folgende Anleitung verwendet werden:\\
\texttt{https://www.pyimagesearch.com/2018/05/28/ubuntu-18-04-how-to-install-opencv/}
Zusätzlich zur Installation von OpenCV muss Flask mit pip installiert werden.
Hierzu muss wie in der Anleitung beschrieben .bashrc mit 
\begin{lstlisting}[frame=single]
source  ~/.bashrc
\end{lstlisting}
geladen werden.
Anschließend kann mit
\begin{lstlisting}[frame=single]
workon cv
\end{lstlisting} in der Virtuellen Python-Umgebung gearbeitet bzw. flask wie folgt installiert werden:
\begin{lstlisting}[frame=single]
pip install flask flask-cors
\end{lstlisting}
Nun muss nur noch das Repository ausgecheckt und ausgeführt werden.
\begin{lstlisting}[frame=single]
git clone https://github.com/PIPCO-1819/PIPCO-Backend
cd PIPCO-Backend
python Main.py
\end{lstlisting}

\section{Frontend}
Für das Frontend wird Node.js, npm, sowie Angular verwendet.
\begin{lstlisting}[frame=single]
apt-get install npm nodejs
sudo npm install -g npm@latest
sudo npm install -g @angular/cli
\end{lstlisting}
Repository auschecken und restliche dependencies installieren:
\begin{lstlisting}[frame=single]
git clone https://github.com/PIPCO-1819/PIPCO-Frontend.git
cd PIPCO-Frontend
npm install
\end{lstlisting}
Anschließend den Server wie folgt starten:
\begin{lstlisting}[frame=single]
ng serve --host 0.0.0.0
\end{lstlisting}

\section{Run on Startup}
start\_backend.sh in PIPCO-Backend
\begin{lstlisting}[frame=single]
#!/bin/sh
printf "<LOGIN>\n<PASSWORD>\n" | \
/home/osboxes/.virtualenvs/cv/bin/python Main.py
\end{lstlisting}

start\_frontend.sh in PIPCO-Frontend
\begin{lstlisting}[frame=single]
#!/bin/sh
printf "<LOGIN>\n<PASSWORD>\n" | \
/home/osboxes/.virtualenvs/cv/bin/python Main.py
\end{lstlisting}

start\_frontend.sh in PIPCO-Frontend
\begin{lstlisting}[frame=single]
#!/bin/sh
ng serve --host 0.0.0.0
\end{lstlisting}

start\_pipco.sh
\begin{lstlisting}[frame=single]
#!/bin/bash
screen -dmS frontend bash -c \
'cd /home/osboxes/PIPCO-Frontend; ./start_frontend.sh'
screen -dmS backend bash -c \
'cd /home/osboxes/PIPCO-Backend;. start_backend.sh'
\end{lstlisting}

rc.local bei Start des Systems ausführen
\begin{lstlisting}[frame=single]
printf '%s\n' '#!/bin/bash' 'exit 0' | sudo tee -a /etc/rc.local
sudo chmod +x /etc/rc.local
\end{lstlisting}
Skript zu rc.local hinzufügen
\begin{lstlisting}[frame=single]
...
/home/osboxes/start_pipco.sh
exit 0
\end{lstlisting}