\chapter{Ausblick}
Eine einfache Bewegungserkennung und Aufnahme der Bewegung wurde erfolgreich umgesetzt. Des weiteren können bereits einige Einstellungen vorgenommen werden. Jedoch bietet dieses Projekt noch viele Möglichkeiten.\\
Die Berechnung des Medians kann in Zukunft auf mehrere Prozesse aufgeteilt werden oder es findet sich eine Möglichkeit die Erkennung der Bewegung über mehrere Frames einfacher zu gestalten.
Mit einer besseren Performance könnten somit auch höhere Auflösungen verarbeitet werden.
Außerdem wäre es praktisch mehrere Kamerastreams mit Tags (z.B. Wohnzimmer) der Website hinzuzufügen zu können und bei einer erkannten Bewegung zusätzlich darüber benachrichtigt zu werden, von welcher Kamera diese Bewegung erkannt wurde. Hierfür müsste ein eigener Prozess für jede Bildverarbeitunsroutine gestartet werden, wobei dann für die Datenhaltung auch eine neue Lösung (IPC) gefunden werden müsste, da diese Prozesse keinen gemeinsamen Bereich im RAM verwenden.\\
Eine weitere Verbesserung wäre die Kommunikation über den Front-End-Server zu leiten, welcher dann vom Backend über Änderungen informiert wird und diese aktualisierten Informationen an den Client weitergibt. Somit wäre das Back-End entlastet und könnte nach außen unsichtbar gemacht werden. Der Login ist momentan außerdem durch die jetzige Implementierung mit etwas Javascript-Änderungen zu umgehen. Es wird lediglich Client-seitig die Antwort des Back-Ends überprüft.\\
Die Benachrichtung könnte man vielleicht auch etwas moderner gestalten. Hierbei wäre eine App oder eine Progressive Web App für eine Push-Benachrichtigung interessant.\\
Mit OpenCV könnte auch in Richtung des Deep Learnings gearbeitet werden und damit vielleicht z.B. das Haustier mittels Object Detection erkannt und somit als Bewegung ignoriert werden.