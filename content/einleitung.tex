\chapter{Einleitung}
\section{Rahmenbedinungen}
Dieses Projekt stellt das Semesterprojekt von vier Studenten des Studienganges Allgemeine Informatik der Hochschule in Furtwangen dar. Es handelt sich dabei um das zweite Semesterprojekt, welches im sechsten Semester stattfindet. 

Ziel des Projektes ist es, eine Software zur Überwachung mittels \acs{IP}-Kamera zu implementieren, wobei die genutzte Hardware austauschbar bleiben soll. Die Anwendung soll die Fähigkeit besitzen, Bewegungen im Kameralivestream zu detektieren und zuvor hinterlegte Nutzer per E-Mail über die erkannten Bewegungen in Kenntnis zu setzen. Außerdem sollen Aufnahmen dieser Bewegungen erstellt und für den Endanwender einsehbar hinterlegt werden. Neben diversen Einstellungsmöglichkeiten für den Nutzer, wie zum Beispiel für die Sensitivität der Bewegungserkennung oder einer maximalen Anzahl an gespeicherten Aufnahmen soll die Anwendung über eine benutzerfreundliche Weboberfläche mit Login-Maske verfügen.

Unter der Betreuung von Prof. Dr. Elmar Cochlovius und Judith Jakob wurde das Projekt weitestgehend selbstorganisiert durchgeführt. Ein für Testzwecke erforderlicher Hardware-Aufbau konnte im Smart-Home-Labor am Campus in Furtwangen genutzt werden. Dort waren auch ähnliche Lösungen von kommerziellen Anbietern vorhanden, welche während dem Projekt als Referenzen gedient haben.